%%%% OPTION
%% Change class according to your needs
%%  - article (no chapter)
%%  - report
%%  - etc.
\documentclass{article}


\IfFileExists{ifxetex.sty}{%
  \usepackage{ifxetex}%
}{%
  \newif\ifxetex
  \xetexfalse
}
  \ifxetex

\usepackage{fontspec}
\usepackage{xltxtra}
\setmainfont{DejaVu Serif}
\setsansfont{DejaVu Sans}
\setmonofont{DejaVu Sans Mono}
\else
\usepackage[T1]{fontenc}
\usepackage[utf8]{inputenc}
\fi
\usepackage{fancybox}
\usepackage{makeidx}
\usepackage{cmap}
\usepackage{url}
\usepackage{eurosym}

\usepackage[hyperlink]{sysfera}



%%%%
%% TODO:
%%  - ajouter une macro pour mettre l'objet du document
%%  - faire les footers avec l'adresse et tels de SysFera
%%  - enlever un max de paquets docbook


%%%%%%%%%%%%%%%%%%%%%%%%%%%%%%%%%%%%%%%%%%%%%%%%%%%%%%%%%%%%%%%%%%%%%%
%                            CONFIGURATION                           %
%%%%%%%%%%%%%%%%%%%%%%%%%%%%%%%%%%%%%%%%%%%%%%%%%%%%%%%%%%%%%%%%%%%%%%
% Use the following macros to configure your document
%%%%%%%%%%%%%%%%%%%%%%%%%%%%%%%%%%%%%%%%%%%%%%%%%%%%%%%%%%%%%%%%%%%%%%
%%%% OPTION
%% Change language: fr/en
%%  - for French: use french in babel, and fr in \setupsysferalocale
%%  - for English: use english in babel, and en in \setupsysferalocale

\usepackage[french]{babel}
\setupsysferalocale{fr}
\frenchbsetup{CompactItemize=false} % Fix itemize clash 
\usepackage{enumitem}
%%%% OPTION
%% Title and author of the document
\title{Rapport d'installation de VISHNU 2.0.0 beta}
\author{K. Coulomb}

%%%% OPTION
%% Document reference
%% Use command \SFdocumentreference to set the document reference.
%% Latter on, you can use the \SFthisdocument macro to retrieve
%% this reference.
\SFdocumentreference{20120919\_CR\_INSTALL\_VISHNU2.0.0}
\SFprojectname{VISHNU}
\SFprojectleader{E.P. Capo-Chichi}
\SFclient{EDF}


%%%% OPTION
%% Release information. If the argument is not empty, then a box with
%% the content of the argument will be visible at the top of the document
%\SFreleaseinfo{Réalisé} % will show "Travail en cours" at the
                                 % top of the page
\SFreleaseinfo{} % won't show anything

%%%% OPTION
%% Draft watermark. You can also show a grey watermak on all pages of
%% your document using the following command.
%\showwatermark{DRAFT}


%%%% OPTION
%% Add a logo into the header.
%% First parameter sets the width of the image relatively to the
%% \textwidth
%% Second parameter is the path to the image
% \SFsetheadlogo{.25}{fig/logosysfera.pdf}

%%%% OPTION
%% Collaborators:
%% You can redefine the Indexation of the document using the following command:
\renewcommand{\SFindexation}{
  \begin{SFindtable}
    \SFinditem{\writtenby}{K. Coulomb}{19 septembre 2012}
%    \SFinditem{\verifiedby}{E.P. CapoChichi}{06 avril 2012}
%    \SFinditem{\approvedby}{E.P. CapoChichi}{06 avril 2012}
  \end{SFindtable}
}
% \renewcommand{\SFindexation}{} % disable this table


%%%% OPTION
%% Revision History Table:
%% Add a new \SFrevitem entry for adding a new entry in the revision
%% history table.
\renewcommand{\SFrevhistory}{
\begin{SFrevtable}
  \SFrevitem{1}{19/09/2012}{Premiere version}{K. Coulomb}
  \SFrevitem{1}{26/09/2012}{Ajout des tests de validation du WebBoard}{Philippe Martinez} 
\end{SFrevtable}
}
% \renewcommand{\SFrevhistory}{} % disable this table


%%%% OPTION
%% References Table
%% Add a new \SFrefitem entry for adding a new entry in the list of
%% reference documents.
%\renewcommand{\SFreferenceTable}{
%\begin{SFreftable}
%  \SFrefitem{ref1}{techDocument}{Ce document}
%  \SFrefitem{ref2}{techDocument}{Ce document}
%\end{SFreftable}
%}
\renewcommand{\SFreferenceTable}{} % disable this table

%%%% OPTION
%% Authorization Table
%% Add a new \SFauthviewitem entry for adding a new entry in the list of
%% authorized users.
\renewcommand{\SFauthview}{
\begin{SFauthviewtable}
  \SFauthviewitem{SysFera}{Tous les membres de SysFera}
  \SFauthviewitem{EDF}{S. Kortas, D. Bateman}
\end{SFauthviewtable}
}
% \renewcommand{\SFauthview}{} % disable this table
%%%%%%%%%%%%%%%%%%%%%%%%%%%%%%%%%%%%%%%%%%%%%%%%%%%%%%%%%%%%%%%%%%%%%%
%                           /CONFIGURATION                           %
%%%%%%%%%%%%%%%%%%%%%%%%%%%%%%%%%%%%%%%%%%%%%%%%%%%%%%%%%%%%%%%%%%%%%%


\makeindex
\makeglossary


\begin{document}

\frontmatter % do not disable, this is used for page numbering
\maketitle % do not disable, otherwise you won't have any title
%%%% OPTION
\tableofcontents % comment to disable the table of contents
\mainmatter % do not disable, this is used for page numbering


%%%%%%%%%%%%%%%%%%%%%%%%%%%%%%%%%%%%%%%%%%%%%%%%%%%%%%%%%%%%%%%%%%%%%%
%              Write your document below this comment                %
%%%%%%%%%%%%%%%%%%%%%%%%%%%%%%%%%%%%%%%%%%%%%%%%%%%%%%%%%%%%%%%%%%%%%%



\section{Pr\'esentation}
    
\subsection{Objectif}
L'objectif de ce document est de faire un compte rendu de la mise à
jour de VISHNU pour sa version 2.0.0 beta ainsi que la mise à disposition du WebBoard sur les machines
de la plateforme de EDF R\&D. Cette mise à jour a été 
réalisée à distance en utilisant le VPN par K. Coulomb et P. Martinez pour SysFera.

\subsection{Plateforme}
A la base, cette mise à jour devait \^etre r\'ealis\'ee sur la 
plateforme de tests de EDF. Toutefois, \`a cause d'un probl\`eme
r\'eseau sur une partie des machines (le r\'eseau HPSLAB \'etait
inaccessible), il n'\'etait pas possible d'installer sur la
plateforme de tests. De fait, les machines sur lesquelles l'installation
a \'et\'e r\'ealis\'ee correspondent aux machines utilis\'ees en production. L'\'el\'ement qui diff\`ere est que sur la plupart
des machines, un compte unix diff\'erent contenant les installations est utilis\'e. Dans un second temps, un déploiement
utilisant les m\^emes comptes unix que la production a \'et\'e 
utilis\'e.

\subsection{Les machines sur comptes secondaires}
\begin{itemize}
\item claui2q1 (login vishnu2) : UMS, FMS, IMS, database, clients
\item clamart2 (login kortas) : TMS de type torque/maui
\item bgp (login kortas) : TMS de type LoadLeveler
\item ivanoe (login : C70353) : TMS de type  Slurm
\item aster4.cla.edfgdf.fr (login kortas) : TMS de type  LSF
\item webboard (login vishnu) : Le webboard, clients VISHNU avec python/java
\end{itemize}

\subsection{Les machines sur comptes production}
\begin{itemize}
\item claui2q1 (login vishnu) : UMS, FMS, IMS, database
\item clamart2 (login vishnu) : TMS de type torque/maui
\item bgp (login vishnu) : TMS de type LoadLeveler
\item ivanoe (login : vishnu) : TMS de type  Slurm
\item aster4.cla.edfgdf.fr (login kortas) : TMS de type  LSF
\end{itemize}

\section{Phase 1}
\subsection{But}
Lors de la premi\`ere phase de l'installation, vu que les
machines de la plateforme de tests \'etaient inaccessibles,
il a \'et\'e décidé de r\'ealiser l'installation sur les machines de la
plateforme de production. Toutefois, pour \'eviter des
conflits, cette installation a \'et\'e faite sur des comptes
unix correspondants \`a d'autres utilisateurs que ceux 
qui ont la version de production install\'ee (voir le d\'etail
des machines avec les comptes secondaires). Le d\'eploiement
de cette installation devra utiliser le m\^eme bus CORBA 
que la version tournant actuellement. 

\subsection{R\'ealisations}
Vu que les \'el\'ements tels DIET, omniORB, boost \'etaient 
d\'ej\`a install\'es sur les machines, ces \'el\'ements l\`a
ont \'et\'e r\'eutilis\'es et non r\'einstall\'es. Il avait \'et\'e pr\'evu
de mettre la derni\`ere version de DIET (2.8.1), mais vu que l'installation
se d\'eroulait sur la plateforme de production,
SysFera a jug\'e pr\'ef\'erable de ne pas installer la
version de DIET la plus \`a jour car SysFera n'\'etait pas confiant
quand \`a la bonne tenue de 2 versions de DIET diff\'erentes sur un 
m\^eme bus CORBA. De plus, il fallait installer sur chaque machine
une version de cmake plus r\'ecente car les besoins de VISHNU pour 
la version 2.0.0 beta ont \'et\'e augment\'e. La version choisie a \'et\'e
la release actuelle de cmake (2.8.9).

\subsection{Probl\`emes rencontr\'es}
Lors de cette premi\`eres phases, divers probl\`emes ont \'et\'e rencontr\'es.
Tout d'abord, sur la machine ivanoe, la phase de compilation de cmake ne 
fonctionnait pas. Pour \^etre pr\'ecis c'est la phase de configure qui bloquait
syst\'ematiquement. Diff\'erentes versions de cmake ont \'et\'e essay\'ees (2.8.6,
2.8.8, 2.8.9) sans succ\`es. En d\'esespoir de cause, les binaires
pr\'ecompil\'es de cmake ont \'et\'e transf\'er\'es sur la machine et
cela a permis de compiler VISHNU. A noter que diff\'erentes options du
configure ont \'et\'e test\'ees (activer/d\'esactiver certains \'el\'ements)
mais cela ne changeait pas le probl\`eme. Une fois ce probl\`eme r\'egl\'e,
la compilation de VISHNU s'est faite sans probl\`eme sur la machine.
L'autre probl\`eme majeur est li\'e \`a la machine BGP. La compilation de 
cmake a r\'eussi mais celle de VISHNU \'echouait avec des probl\`emes d'include
directory. Apr\`es quelques petits tests pour essayer de r\'esoudre
le probl\`eme, une discussion avec EDF a mis en avant le fait que la machine
allait bient\^ot \^etre arr\'et\'ee et qu'il ne fallait pas perdre
de temps avec elle, de fait les efforts de r\'esolution de probl\`emes
sont pass\'es sur le probl\`eme cmake sur ivanoe.
Un autre probl\`eme a \'et\'e le fait de garder la cohérence au niveau du modèle de données entre
VISHNU et le WebBoard. En effet, la restauration n'était pas possible lorsque des informations du WebBoard étaient enregistrées. SysFera
a donc mis à disposition un patch permettant de restaurer une configuration sans blocage au niveau du modèle de données. Ce patch impacte uniquement le cot\'e serveur et il n'a pas \'et\'e
appliqu\'e sur les autres machines car UMS n'est pas destin\'e \`a tourner 
dessus. Il est important de noter que la validation de ce patch a \'et\'e freiné par un probl\`eme d'espace disque sur la machine (No space left on device).

\section{Phase 2}
\subsection{But}
La seconde phase est bas\'ee sur la nouvelle version de DIET (2.8.1)
qui a \'et\'e install\'ee par Mr Kortas sur les machines de 
production sur les comptes unix de production. Le déploiement a utilis\'e
un autre port et d'autres instances du bus de communication (CORBA)
afin d'\'eviter au maximums d'\'eventuels conflits.

\subsection{Probl\`emes}
Les versions de VISHNU qui tournent d\'ej\`a sur les machines utilisent
la version de DIET 2.8.0 et la hi\'erarchie est d\'ej\`a d\'eploy\'ee.
Toutefois, lors du d\'eploiement de VISHNU 2.0.0 beta, malgré le 
l'utilisation de CORBA sur un autre port, le démarrage des 
\'el\'ements au travers des forwarders ne se faisait pas correctement.
La source exacte du probl\`eme n'a pas \'et\'e identifi\'ee, mais 
SysFera suppose que la pr\'esence de deux binaires forwarders diff\'erents
repr\'esentants des tunnels entre les deux m\^emes machines est la
cause du probl\`eme, le mauvais forwarder devant \^etre sollicit\'e 
ceci devant entrainer un blocage au niveau de CORBA avec des objets 
similaires diff\'erents entre la source en DIET 2.8.1 et la destination
en DIET 2.8.0 (ou vice versa). 
Pour valider cette source d'erreur, la plateforme de production a 
\'et\'e coup\'ee, et cela laissait l'installation 2.0.0 fonctionner.
SysFera a donc conclu que deux instances diff\'erentes de DIET ne peuvent 
pas fonctionner simmultan\'ement en utilisant le m\'echanisme des
forwarders.

\section{Remarque g\'en\'erales}
Contrairement \`a d'habitude, SysFera a du faire plusieurs 
transferts de donn\'ees plut\^ot volumineux, avec la mise en
place d'une nouvelle connexion en fibre. De ce fait, les 
transferts \'etaient plus rapides et cette rapidit\'e des 
transferts entrainait un crash du VPN. SysFera a donc
contact\'e le support li\'e au VPN et un rapport d'incident 
a \'et\'e ouvert. D'une mani\`ere g\'en\'erale, les connexions sautent
de mani\`ere al\'eatoire, plus ou moins fr\'equement selon les jours.

\section{Tests de validation}

Les difficultés d'accès au VPN n'ont pas permis d'effectuer une liste exhaustive de tests. Des services repr\'esentatifs des modules 
sur chaque machine ont \'et\'e manuellement appell\'es 
(soumission de job et liste des queues sur TMS par exemple ou 
connexion et liste des machines sur UMS). Vu que FMS et IMS ne sont
pas impact\'ees par les changements pour la release 2.0.0, les tests
les concernant ont \'et\'e limit\'es.
Ci-dessous les divers sc\'enarios test\'es~:

\subsection{Connexion avec netrc}
\begin{itemize}
\item Lancement de la connexion sans préciser les identifiants\\
  \textit{vishnu\_connect}
\end{itemize}

\subsection{Connexion avec nom d'utilisateur/mot de passe}
\begin{itemize}
\item Lancement de la connexion avec nom d'utilisateur et mot de passe\\
  \textit{vishnu\_connect -u root -w vishnu\_user}
\end{itemize}

\subsection{Soumission d'un job sur aster4}
\begin{itemize}
\item Lancement de la connexion sans préciser les identifiants\\
  \textit{vishnu\_connect}
\item Soumission d'un job\\
  \textit{vishnu\_submit\_job aster4 $\sim$/SYSFERA/src/TMS/test/src/script/fast\_lsf\_script} \\
  \textit{Note : le script r\'eellement utilis\'e est l\'eg\`erement modifi\'e pour correspondre aux exigences de lsf sur aster4} \\
\end{itemize}

\subsection{Soumission d'un job sur ivanoe}
\begin{itemize}
\item Lancement de la connexion sans préciser les identifiants\\
  \textit{vishnu\_connect}
\item Soumission d'un job\\
  \textit{vishnu\_submit\_job ivanoe $\sim$/SYSFERA/src/TMS/test/src/script/fast\_slurm\_script}\\
  \textit{Note : le script r\'eellement utilis\'e est l\'eg\`erement modifi\'e pour correspondre aux exigences de SLURM sur ivanoe} \\
\end{itemize}

\subsection{Soumission d'un job sur clamart2}
\begin{itemize}
\item Lancement de la connexion sans préciser les identifiants\\
  \textit{vishnu\_connect}
\item Soumission d'un job \\
  \textit{vishnu\_submit\_job clamart2 $\sim$/SYSFERA/src/TMS/test/src/script/fast\_torque\_script}\\
  \textit{Note : le script r\'eellement utilis\'e est l\'eg\`erement modifi\'e pour correspondre aux exigences de TORQUE sur clamart2} \\
\end{itemize}

\subsection{Lister les queues sur clamart2}
\begin{itemize}
\item Lancement de la connexion sans préciser les identifiants\\
  \textit{vishnu\_connect}
\item Soumission d'un job\\
  \textit{vishnu\_list\_queues clamart2} \\
\end{itemize}

\subsection{Lister les queues sur clamart2}
\begin{itemize}
\item Lancement de la connexion sans préciser les identifiants\\
  \textit{vishnu\_connect}
\item Soumission d'un job\\
  \textit{vishnu\_list\_queues clamart2} \\
\end{itemize}

\subsection{Lister les queues sur clamart2}
\begin{itemize}
\item Lancement de la connexion sans préciser les identifiants\\
  \textit{vishnu\_connect}
\item Lister les queues\\
  \textit{vishnu\_list\_queues clamart2} \\
\end{itemize}

\subsection{Lister les queues sur aster4}
\begin{itemize}
\item Lancement de la connexion sans préciser les identifiants\\
  \textit{vishnu\_connect}
\item Lister les queues\\
  \textit{vishnu\_list\_queues aster4} \\
\end{itemize}

\subsection{Lister les queues sur ivanoe}
\begin{itemize}
\item Lancement de la connexion sans préciser les identifiants\\
  \textit{vishnu\_connect}
\item Lister les queues\\
  \textit{vishnu\_list\_queues ivanoe} \\
\end{itemize}

\subsection{Lister le contenu d'un r\'epertoire sur aster4}
\begin{itemize}
\item Lancement de la connexion sans préciser les identifiants\\
  \textit{vishnu\_connect}
\item Lister le r\'epertoire /tmp\\
  \textit{vishnu\_list\_dir aster4:/tmp}
\end{itemize}

\subsection{Bilan r\'ecapitulatif}

\begin{tabular}{|c|c|}
\hline 
Fonctionnalités & Résultats \\
\hline 
Connexion netrc & OK \\
\hline
Connexion sans netrc & OK \\
\hline
Soumission aster4 & OK \\
\hline
Liste queues aster4 & OK \\
\hline
Soumission ivanoe & OK \\
\hline
Liste queue ivanoe & OK \\
\hline
Soumission clamart2 & OK \\
\hline
Liste queues clamart2 & OK \\
\hline
Lister le tmp de clamart2 & OK \\
\hline
\end{tabular}

\section{Installation du WebBoard}

Dans cette section, nous allons décrire les différents tests qui ont été effectués pour valider le fonctionnement du WebBoard dans 
les locaux d'EDF. Il est important de noter que cette installation s'est faite à distance via le VPN et que de nombreux problèmes de 
connexion ont été rencontrés ce qui a rendu difficile non seulement la phase d'installation mais aussi de validation.

\begin{tabular}{|c|c|}
\hline 
Fonctionnalités & Résultats \\
\hline 
Connexion au WebBoard via un compte administrateur & OK \\
\hline
Connexion au WebBoard via un compte utilisateur & OK \\
\hline
Création d'un utilisateur & OK \\
\hline
Création d'un work & OK \\
\hline
Création d'un projet & OK \\
\hline
Soumission d'un job sur ivanoe & OK \\
\hline
Configuration d'une application & OK \\
\hline
Utilisation du browser de FMS & OK \\
\hline
Vérification de l'envoi d'emails & A valider par EDF \\
\hline
Récupération de fichiers de sortie associés aux jobs & Pas OK \\
\hline
\end{tabular}


%%%%%%%%%%%%%%%%%%%%%%%%%%%%%%%%%%%%%%%%%%%%%%%%%%%%%%%%%%%%%%%%%%%%%%
%                            BIBLIOGRAPHY                            %
%%%%%%%%%%%%%%%%%%%%%%%%%%%%%%%%%%%%%%%%%%%%%%%%%%%%%%%%%%%%%%%%%%%%%%
%%%% OPTION
%% You can deactivate bibliography by commenting the following 3
%% lines.
%% If you wish to add a bibliography, you need to update the
%% \biblography command. Its argument is the name of .bib file.
%%\clearpage
%\bibliography{techDocument}
%\bibliographystyle{plain}
%%%%%%%%%%%%%%%%%%%%%%%%%%%%%%%%%%%%%%%%%%%%%%%%%%%%%%%%%%%%%%%%%%%%%%
%                           /BIBLIOGRAPHY                            %
%%%%%%%%%%%%%%%%%%%%%%%%%%%%%%%%%%%%%%%%%%%%%%%%%%%%%%%%%%%%%%%%%%%%%%
    
\end{document}
